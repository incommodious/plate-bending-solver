\documentclass[11pt]{article}
\usepackage{amsmath,amssymb}
\usepackage{geometry}
\usepackage{booktabs}
\usepackage{graphicx}
\usepackage{siunitx}
\usepackage{float}
\geometry{margin=1in}
\title{Rectangular Plate Bending Analysis\\[0.5em]\large 6'' $\times$ 1.5'' FCFC Steel Plate --- Circular Patch Load}
\author{Plate Bending Solver (Rayleigh--Ritz Method)}
\date{February 8, 2026}
\begin{document}
\maketitle

\section{Plate Geometry \& Setup}

\begin{figure}[H]
\centering
\includegraphics[width=0.85\textwidth]{fig_geometry.png}
\caption{Plate geometry, boundary conditions, and applied load.}
\end{figure}

\subsection{Geometry \& Material}
\begin{tabular}{lll}
\toprule
Parameter & Symbol & Value \\
\midrule
Plate length (x-direction) & $a$ & 6.000 in \\
Plate width (y-direction) & $b$ & 1.500 in \\
Thickness & $h$ & 0.070 in \\
Young's modulus & $E$ & $29 \times 10^{6}$ psi \\
Poisson's ratio & $\nu$ & 0.3 \\
\bottomrule
\end{tabular}

\subsection{Flexural Rigidity}
\begin{equation}
D = \frac{Eh^3}{12(1-\nu^2)} = \frac{29 \times 10^{6} \times 0.070^3}{12(1-0.3^2)} = 910.9 \ \text{lbf\,\textperiodcentered\,in}
\end{equation}

\subsection{Boundary Conditions}
Configuration: \textbf{FCFC}
\begin{itemize}
\item $x = 0$ (short edge): \textbf{Free}
\item $y = 0$ (long edge): \textbf{Clamped}
\item $x = a$ (short edge): \textbf{Free}
\item $y = b$ (long edge): \textbf{Clamped}
\end{itemize}

\subsection{Loading}
Concentrated force $P = 68$ lbf distributed over a $\varnothing 0.7''$ circular area ($R = 0.35''$), centered at $(x_0, y_0) = (0.75,\, 0.75)$ in.

\begin{equation}
q_0 = \frac{P}{\pi R^2} = \frac{68}{\pi \times 0.35^2} = 176.7 \ \text{psi}
\end{equation}

\section{Governing Equation}
\begin{equation}
D\nabla^4 w = D\!\left(\frac{\partial^4 w}{\partial x^4} + 2\frac{\partial^4 w}{\partial x^2 \partial y^2} + \frac{\partial^4 w}{\partial y^4}\right) = q(x,y)
\end{equation}

\section{Solution Method --- Rayleigh--Ritz}
FCFC is non-Levy (x-edges are free, not simply supported), so the Rayleigh--Ritz method is used:
\begin{equation}
w(x,y) = \sum_{m=1}^{M}\sum_{n=1}^{N} A_{mn}\,\phi_m\!\left(\frac{x}{a}\right)\,\psi_n\!\left(\frac{y}{b}\right)
\end{equation}
with Free--Free eigenfunctions $\phi_m$ in $x$ and Clamped--Clamped eigenfunctions $\psi_n$ in $y$.
Final solution uses $M = N = 15$ (225 DOFs).

\section{Results at Load Center}

$(x, y) = (0.75,\, 0.75)$ in:

\subsection{Deflection \& Internal Moments}
\begin{tabular}{llrl}
\toprule
Quantity & Symbol & Value & Units \\
\midrule
Deflection & $w$ & 0.000962 & in (0.96 mil) \\
Bending moment (x) & $M_x$ & 218.6 & lbf\,\textperiodcentered\,in/in \\
Bending moment (y) & $M_y$ & 308.9 & lbf\,\textperiodcentered\,in/in \\
\bottomrule
\end{tabular}

\subsection{Bending Stresses}
\begin{align}
\sigma_x &= \frac{6 M_x}{h^2} = \frac{6 \times 218.6}{0.070^2} = 6800 \ \text{psi} \ (6.80 \ \text{ksi}) \\
\sigma_y &= \frac{6 M_y}{h^2} = \frac{6 \times 308.9}{0.070^2} = 9607 \ \text{psi} \ (9.61 \ \text{ksi})
\end{align}

\subsection{Global Maxima}
\begin{tabular}{llrl}
\toprule
Quantity & & Value & Units \\
\midrule
Max deflection & $|w|_{\max}$ & 0.000964 & in \\
Max $|\sigma_x|$ & & 6877 & psi (6.88 ksi) \\
Max $|\sigma_y|$ & & 11303 & psi (11.30 ksi) \\
\bottomrule
\end{tabular}

\section{Deflection Contour}
\begin{figure}[H]
\centering
\includegraphics[width=0.95\textwidth]{fig_deflection.png}
\caption{Deflection field (in mils). Dashed circle marks the loaded area.}
\end{figure}

\section{Stress Contours}
\begin{figure}[H]
\centering
\includegraphics[width=0.95\textwidth]{fig_stress.png}
\caption{Bending stress fields $\sigma_x$ and $\sigma_y$ (ksi).}
\end{figure}

\section{Deflection Profile}
\begin{figure}[H]
\centering
\includegraphics[width=0.85\textwidth]{fig_profile.png}
\caption{Deflection along the plate centerline ($y = 0.75''$).}
\end{figure}

\section*{Notes}
\begin{itemize}
\item Material: structural steel ($E = 29$ Msi, $\nu = 0.3$).
\item Kirchhoff thin plate theory ($h/b = 0.047 \ll 1$).
\item Stresses are max surface values at $z = \pm h/2$.
\end{itemize}

\appendix
\section{Step-by-Step Ritz Calculation}
\label{app:worked}

This appendix walks through the Ritz solution in detail, showing every number so the calculation can be followed or reproduced.

\subsection{Setup}

\begin{tabular}{lrl}
\toprule
Quantity & Value & \\
\midrule
$a$ (plate length) & $6$ & in \\
$b$ (plate width) & $1.5$ & in \\
$h$ (thickness) & $0.07$ & in \\
$E$ (Young's modulus) & $29 \times 10^{6}$ & psi \\
$\nu$ (Poisson's ratio) & $0.3$ & \\
$q_0$ (pressure) & $176.7$ & psi \\
\bottomrule
\end{tabular}

\medskip\noindent\textbf{Flexural rigidity:}
\begin{align}
D &= \frac{Eh^3}{12(1-\nu^2)} \notag \\
  &= \frac{29 \times 10^{6} \times (0.07)^3}{12(1-0.3^2)} \notag \\
  &= \frac{29 \times 10^{6} \times 343 \times 10^{-6}}{10.92} \notag \\
  &= 910.9 \ \text{lbf\,\textperiodcentered\,in}
\end{align}

\subsection{Trial Function}

The deflection is approximated as a double sum of beam eigenfunctions:
\begin{equation}
w(x,y) = \sum_{m=1}^{M}\sum_{n=1}^{N} A_{mn}\,\phi_m\!\left(\frac{x}{a}\right)\,\psi_n\!\left(\frac{y}{b}\right)
\end{equation}

\begin{itemize}
\item $\phi_m(\xi)$: \textbf{FF} (Free--Free) beam eigenfunctions in $x$
\item $\psi_n(\eta)$: \textbf{CC} (Clamped--Clamped) beam eigenfunctions in $y$
\end{itemize}

Minimizing the total potential energy $\Pi = U - W_{\text{ext}}$ gives:
\begin{equation}
\mathbf{K}\,\mathbf{A} = \mathbf{F}
\end{equation}
where $\mathbf{K}$ is the stiffness matrix (from plate strain energy) and $\mathbf{F}$ is the load vector.

\subsection{Convergence Study}

\noindent The series is evaluated at increasing truncation levels. At each level, the deflection at the analysis point $(0.75,\,0.75)$ in is recorded.

\noindent $M = N = 2$: solve $\rightarrow$ $w = 0.4987$ mil (first evaluation).

\noindent $M = N = 6$: $w = 0.744$ mil. $|\Delta w| = 0.245$ mil.

\noindent Continuing to $M = N = 15$: $w = 0.9618$ mil, $|\Delta w| = 0.0153$ mil (1.6\%) --- converged.

\begin{tabular}{rrrr}
\toprule
$M = N$ & DOFs & $w$ (mil) & $|\Delta w|$ (mil) \\
\midrule
2 & 4 & $0.4987$ & --- \\
6 & 36 & $0.744$ & $0.245$ (33\%) \\
8 & 64 & $0.9029$ & $0.159$ (18\%) \\
10 & 100 & $0.9431$ & $0.0403$ (4\%) \\
12 & 144 & $0.9464$ & $3.27 \times 10^{-3}$ (0.3\%) \\
\textbf{15} & \textbf{225} & $\mathbf{0.9618}$ & $0.0153$ (2\%) \\
\bottomrule
\end{tabular}

\subsection{Physical Intuition}

\textit{This plate has 2 clamped edges and 2 free edges. 
The clamped edges provide strong rotational restraint, limiting deflection but 
concentrating bending moments near the supports. 
The free edge(s) allow the plate to deflect without constraint, so maximum 
deflection typically occurs near or at the free edges. 
The high aspect ratio (4:1) means the plate behaves 
somewhat like a wide beam in the short direction. 
}


\section*{References}
\begin{enumerate}
\item Timoshenko, S. and Woinowsky-Krieger, S., \emph{Theory of Plates and Shells}, 2nd ed., 1959.
\item Ventsel, E. and Krauthammer, T., \emph{Thin Plates and Shells}, Marcel Dekker, 2001.
\end{enumerate}

\end{document}
